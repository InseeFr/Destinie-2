\section{Guide de prise en main du modèle Destinie}
\subsection{Lancer une projection avec le modèle Destinie}
Ce guide vise à faciliter la première prise en main du modèle, tant pour des utilisateurs voulant utiliser le modèle dans sa forme actuelle (voir parties \ref{subsec::instmin} et \ref{subsec::simul}), que pour des utilisateurs plus experts voulant accéder au code source.\\
\subsubsection{Installation minimale}
\label{subsec::instmin}
\begin{enumerate}
\item Installations préalables : \\
\begin{itemize}
\item Installer	R (\url{https://cran.r-project.org/}) 
\item Installer	Rtools (\url{https://cran.r-project.org/bin/windows/Rtools/}), si votre système d'exploitation est 				Windows pour pouvoir installer le package destinie à partir des fichiers source. Choisir la version de Rtools compatible 		avec la version de R choisie précédemment. 
\end{itemize}
\item Installations recommandées : \\
\begin{itemize}
\item Installer	Rstudio (\url{https://www.rstudio.com/products/rstudio/download/}), environnement de travail R auquel il est parfois fait référence dans la documentation.
\item Installer git (\url{https://git-scm.com/}) de façon à pouvoir suivre les futures mises à jour du modèle.
\item Installer par exemple TortoiseGit (\url{https://tortoisegit.org/}) pour pouvoir utiliser Git en "clic-bouton".
\item Disposer d'un tableur (tel Microsoft Excel ou LibreOffice Calc).
\end{itemize}
\item Installer les packages nécessaires à Destinie :\\
\begin{lstlisting}
install.packages(c("devtools","pkgbuild"))
\end{lstlisting}
\item Installation du package Destinie :\\
L'installation se faisant à partir des fichiers sources, il est nécessaire pour les utilisateurs de Windows de lancer au préalable : 
\begin{lstlisting}
devtools::find_rtools()
# ou pkgbuild::find_rtools()
\end{lstlisting} pour obtenir un "TRUE".\\
Puis \begin{lstlisting}
devtools::install_github("InseeFr/Destinie-2",ref="release")
\end{lstlisting} (par défaut les packages dont dépend destinie seront installés)
\end{enumerate}

\subsubsection{Réaliser une première simulation}
\label{subsec::simul}
Dans le logiciel R,
\begin{enumerate}
\item On charge le package : \textit{library(destinie)}
\item On lance une simulation exemple : \textit{demo(simulation,package="destinie",encoding="utf8")} Le fichier source simulation.R du répertoire demo permet de lancer une simulation à partir d'un échantillon test non représentatif \footnote{\underline{Attention :} Le fichier test fourni pour prendre en main le logiciel n'est pas représentatif de la population au 1er janvier 2010. Le fichier représentatif à utiliser est mis à disposition sur la plateforme Quetelet (\url{http://quetelet.progedo.fr/}).} en choisissant des hypothèses classiques.\\
Le déroulé du programme est le suivant :
\begin{lstlisting}
####################################
#echantillon de depart
######################################
library(destinie)
data("test")
simul=test
rm(test)
###################################
#remplacer ici test  par l'echantillon disponible sur Quetelet 
###################################


##############################################
# Champ de la simulation 
##############################################
champ<-"FE" # ou  FM
fin_simul<-2070 #2110 au maximum ou 2070 plus classiquement



####################################
#choix d'options
######################################
    simul$options_salaires <- list()    
    
    simul$options <- list("tp",anLeg=2016,pas1=3/12,pas2=1,
                      AN_MAX=as.integer(fin_simul),champ,
                      NoAccordAgircArrco=F, NoRegUniqAgircArrco=T, 
                      SecondLiq=F,mort_diff_dip=T,effet_hrzn=T) 
                      # pas1=1/12 (pas mensuel pour proj COR)

  ################
  #choix du scenario demographique 
  ################
  # ici la fecondite, l'esperance de vie et le solde migratoire suivent le scenario central des projections de l'Insee
  # pour la France entiere (attention a la coherence avec le champ precedemment choisi)
  # deux autres scenarios sont deja crees le premier ou tous les scenarios sont a bas et ce qui aboutit a une population agee
  # le second tous les scenarios sont a haut et ce qui aboutit a une population jeune    
  # les autres scenarios s'obtiennent en utilisant le programme \data_raw\obtention_hypdemo.R    
  data("fec_Cent_vie_Cent_mig_Cent")
  demo=fec_Cent_vie_Cent_mig_Cent
  rm(fec_Cent_vie_Cent_mig_Cent)

  ############
  #chargement des equations regissant le marche du travail, de sante 
  ################

  data("eq_struct")
  
  ###################
  #chargement des parametres economiques puis projection des parametres dans le futur
  ##################

  data("eco_cho_7_prod13")
  eco=eco_cho_7_prod13
  rm(eco_cho_7_prod13)
  eco$macro <-eco$macro%>%
    mutate( 
      SMPTp = ifelse(is.na(SMPTp),0,SMPTp),
      SMICp = ifelse(is.na(SMICp),0,SMICp),
      PIBp = ifelse(is.na(PIBp),0,PIBp),
      PlafondSSp = ifelse(is.na(PlafondSSp),0,PlafondSSp),
      Prixp = ifelse(is.na(Prixp),0,Prixp),
      MinPRp = 1.02,
      RevaloRG = ifelse(is.na(RevaloRG),1+Prixp,RevaloRG),
      RevaloFP = ifelse(is.na(RevaloFP),1+Prixp,RevaloFP),
      RevaloSPC = ifelse(is.na(RevaloSPC),1+Prixp,RevaloSPC)
    ) %>%
    projection(
      SMPT ~ cumprod((1+SMPTp)*(1+Prixp)),
      PIB ~ cumprod((1+PIBp)*(1+Prixp)),
      PlafondSS ~ cumprod((1+PlafondSSp)*(1+Prixp)),
      SMIC ~ cumprod((1+SMICp)*(1+Prixp)),
      Prix ~ cumprod(1+Prixp),
      PointFP|PlafRevRG ~ SMPT,
      SalValid ~ SMIC,
      PlafARS1|PlafARS2|PlafARS3|PlafARS4|PlafARS5|PlafCF3|PlafCF4|PlafCF5|
        MajoPlafCF|sGMP|BMAF|SeuilPauvrete ~ SMPT,
      MaxRevRG ~ PlafondSS,
      MinPR ~ cumprod(MinPRp*(1+Prixp)),
      MinVieil1|MinVieil2|Mincont1|Mincont2 ~ lag(cumprod(1+Prixp)), # indexation standard. En evolution, indexation sur l'inflation de t-1.
      SalRefAGIRC_ARRCO|SalRefARRCO|SalRefAGIRC ~  cumprod(ifelse(annee%in%c(2016,2017,2018),(1+SMPTp+0.02)*(1+Prixp),(1+SMPTp)*(1+Prixp))),
      ValPtAGIRC|ValPtARRCO|ValPtAGIRC_ARRCO ~ cumprod(1+ifelse(annee%in%c(2016,2017,2018),pmax(Prixp-0.01,0),Prixp)),
      MinRevRG|SeuilExoCSG|SeuilExoCSG2|SeuilTxReduitCSG|SeuilTxReduitCSG2 ~cumprod(1+Prixp),
      .~1
    ) 

  eco$macro=eco$macro%>%filter(annee<=fin_simul)
  eco$CiblesTrans <-  left_join(eco$macro %>% select(annee), eco$CiblesTrans)
####################
# rassemblement dans un unique environnement
#####################
  eco=as.list(eco)
  demo=as.list(demo)
  simul=as.list(simul)
  eq_struct=as.list(eq_struct)
  simulation=as.environment(c(demo,eco,simul,eq_struct))
  rm(demo,eco,simul,eq_struct)
############
#simulation
###########
destinieDemographie(simulation)
destinieTransMdt(simulation)
destinieImputSal(simulation)
destinieCalageSalaires(simulation)
destinieSim(simulation)



##############################################
# Resultats ----------------------------------
#age moyen de liquidation pour tous et par sexes
simulation$Indicateurs_an %>% 
  filter(regime=="tot" & annee > 2000) %>%
  ggplot(aes(x=annee,y=Age_Ret_Flux,color=sexe)) + geom_line()
#masse des pensions sur le Pib
simulation$Indicateurs_an %>%
  filter(regime=="tot"& sexe=="ens" & annee > 2010& annee<=2070)%>%
  left_join(simulation$macro)%>% 
  ggplot(aes(x=annee,y=M_Pensions_ma/10/PIB,color=sexe)) + geom_line()+
  theme_bw()

\end{lstlisting}
\end{enumerate}


\subsection{Pour obtenir/utiliser les sources}

\begin{enumerate}
\item Récupérer le dossier contenant le modèle : 
Créer un dossier Destinie ; puis y clôner le dépôt en le nommant destinie. \footnote{Le fait que le dossier soit dans \url{/Destinie/destinie} permet aux liens dans les fichiers de paramètres de rester corrects.} \\
Par exemple, pour les utilisateurs de TortoiseGit :
\begin{itemize}
\item Faire un clic droit dans l'explorateur
\item Cliquer sur "Git clôner"
\item  dans URL, inscrire : \url{https://github.com/InseeFr/Destinie-2}
\item dans Répertoire inscrire l'emplacement où vous souhaitez installer Destinie. \item Récupérer le dossier contenant le modèle : 
Créer un dossier Destinie ; puis y clôner le dépôt en le nommant destinie. \\
\end{itemize}
%\item Chargement du package devtools. Exécuter: \\
% \url{assignInNamespace("version_info", c(devtools:::version_info, list("3.5" = list(version_min = "3.3.0", version_max = "99.99.99", path = "bin"))), "devtools")}\footnote{Cette ligne est nécessaire au fonctionnement sous R.3.5.0 au 19 juillet 2017 ; ce passage se simplifiera dans quelques semaines, d’après : \url{https://github.com/r-lib/devtools/issues/1772\# issuecomment-388639815}}  \\
% 
% \url{find_rtools( )}  \\
% Un "TRUE" devrait alors s'afficher.
  \item  Installer le package devtools, et dans Rstudio utiliser les commandes de Build pour charger destinie ou compiler le package source .tar.gz ou le windows binairies. 
\end{enumerate}



\underline{En cas de difficultés :}\\
Si vous avez des difficultés liées au modèle que vous n'arrivez pas à résoudre, sur le dépôt public (actuellement sur \url{https://github.com/InseeFr/Destinie-2}), vous trouverez un onglet : "Issues".\\
Vous pouvez rechercher si d'autres personnes ont déjà rencontré ce problème, et si la communauté des utilisateurs y a répondu, notamment en filtrant sur les étiquettes (\textit{label}) \textit{"Good for newcomers"}.\\
Si le problème persiste, vous pouvez y rédiger un message. Déclinez votre nom, puis présentez du mieux possible un exemple minimal, complet et vérifiable (voir quelques conseils sur ce sujet ici: \url{https://stackoverflow.com/help/mcve}). Enfin, étiquetez le nouveau sujet à l'aide par exemple d'un ou de plusieurs des codes suivants :
\begin{itemize}
\item \textit{"Good for newcomers"} (Bien pour les débutants)
\item \textit{"bug"}
\item \textit{"Something isn't working"} (Quelque chose ne marche pas)
\item \textit{"duplicate"} (répétition)
\item \textit{"enhancement"} (amélioration)
\item \textit{"New feature or request"} (Nouvelle fonctionnalité)
\item \textit{"This doesn't seem right"} (Cela ne semble pas correct)
\item \textit{"question"} (Question)
\item \textit{"Further information is requested"} (De plus amples informations sont demandées)\\
\end{itemize}


\underline{A noter :}\\
Au sein du projet lui-même, quelques conseils pratiques très courants sont également intégrés au sein d'un fichier \url{"conseils_pratiques.txt"}.
